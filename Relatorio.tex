\documentclass[12pt,a4paper]{article}


\usepackage{amsfonts}
\usepackage{amssymb}
\usepackage[utf8]{inputenc}
\usepackage[portuguese]{babel}
\usepackage{hyperref}
\usepackage{float}
\usepackage[usenames,dvipsnames]{xcolor}
\usepackage{caption}
\usepackage{subcaption}
\usepackage{hyperref}

\hypersetup {
	colorlinks = true,
	urlcolor = Cerulean
}

\title{\vspace{-130pt}
River Raid - Major Gubi \\
		Relatório}
\author{Ricardo Lira da Fonseca\\Renato Lui Geh\\Yan Soares Couto\\GitHub: \url{https://github.com/yancouto/LabProgGame}}
\date{}

\renewcommand\thesection{\Roman{section} -}
\renewcommand\thesubsection{\arabic{subsection}.}
\renewcommand\thesubsubsection{\roman{subsubsection})}

\begin{document}
	\maketitle
	\section{Primeira Fase}
	\begin{verbatim}
Ground control to Major Gubi: commencing countdown, report go-bi!

                               *     .--.
                                    / /  `
                   +               | |
                          '         \ \__,
                      *          +   '--'  *
                          +   /\
             +              .'  '.   *
                    *      /======\      +
                          ;:.  _   ;
                          |:. (_) G|
                          |:.  _  U|
                +         |:. (_) B|          *
                          ;:.     I;
                        .' \:.    / `.
                       / .-'':._.'`-. \
                       |/    /||\    \|
                     _..--"""````"""--.._
               _.-'``                    ``'-._
             -'                                '-
	\end{verbatim}
		\subsection{Decisões Tomadas}
		Para fazer o trabalho, mesmo na fase inicial, nosso grupo se reuniu para discutir como tratar vários 					aspectos do jogo.
	
	  	Primeiramente utilizamos muitos das idéias básicas que definem o tipo e base do jogo:
	  	\begin{itemize}
	  		\item Uma nave que andaria em 3 dimensões
	  		\item Inimigos que atiram na nave
	  		\item Cenário que vai se deslocando com uma lista de inimigos que se aproximam relativamente à nave
	  	\end{itemize}
	  	Logo no começo foi visto decisões sobre movimentos dos objetos dos jogo. Naves se moveriam
		sempre para frente com uma velocidade que pode variar, e além disso se moveriam lateralmente
		e verticalmente em relação à tela. Entretanto decidimos por praticidade fazer ela temporariamente
		paralela ao ``chão'', ou seja angularmente o que variaria nela seria apenas o inicio do percurso
		dos projéteis atirados. Logo a nave teria como características sua posição em coordenadas no
		Cenário (x, y, z), uma velocidade e os valores de direção da ``arma'' acoplada nela.\\
		
	  	Com essa decisão não teriamos de nos preocupar muito no inicio com colisões ou a nave ``fugir''
		da tela já que nos testes iniciais utilizamos apenas o terminal, sem nenhuma parte gráfica. Por
		essa mesma razão de praticidade por enquanto só mudamos diretamente a posição x e y da nave,
		sem nenhuma aceleração individual nessas coordenadas para o movimento, e não rotacionamos a nave
		para facilitar a colisão.\\
		
	  	Logo a direção ``vetorial'' será mais utilizada no tiros, que ao serem criados, terão sua
		posição e direção definidas pelas iniciais da nave. Tiros tambpem foram decididos inicialmente
		para serem pontos em vez de terem tamanhos 3D como inimigos ou a nave principal (novamente para
		facilitar na hora de colisão nessa etapa de desenvolvimento).\\
		
    		Tiros colidirão tanto com a nave quanto com os inimigos. Uma outra decisão tomada foi que
		tiros que inimigos atiram podem destruir outros inimigos. Isso foi escolhido tanto por razões de
		praticidade nos testes, como seria uma feature bacana no jogo, onde o próprio cenário é sujeito
		a sua auto destruição. Em fases futuras isso será decidido se permanece no projeto final.
    		Desta forma os tiros tem um valor de ``vida'' assim como inimigos e a nave principal. Se por
		acaso ao desenvolver melhor o jogo certos tiros tiverem a habilidade de atravessar inimigos ou
		objetos, a manipulação de sua vida para que não abaixe de 0 instantanemente seria útil.\\
		
    		Sobre o Cenário decidimos da seguinte forma:
    		\begin{itemize}
    			\item O Cenário, chamado de Scene, é composto de partições, ditas Sections, por meio de uma lista
			ligada do tipo FIFO. Essas partições são, por sua vez, compostas de um vetor posição (x, y, z),
			um vetor dimensão (width, height, length) e uma lista de inimigos. Quando o jogador, que controla
			a nave , entra em uma Section, a anterior é destruída, implicando na destruição dos inimigos
			contidos na partição. Em seguida, nós criamos uma nova partição no final da lista de
			partições presente em Scene.
		
    			\item Cada Section, quando criada, constrói um número de inimigos gerado pseudo-aleatoriamente em
			posições diferentes mas que estejam dentro do paralelepípedo gerado por (x, y, z) e
			(width, height, length).	
    		\end{itemize}
    		
    		\subsection{Problemas}
    		O maior problema que tivemos foi ter atrasado muito para começar o projeto, e por culpa disso
		tivemos que acelerar muito pra terminar a tempo, logo não só prejudicando o pensamento de ideias
		boas para o projeto, como não podendo sofisticar a parte de testes e melhorá-lo. Isso será algo
		que mais focaremos em outras fases: Organização.
	
	\newpage
	\section{Segunda Fase}
	\begin{verbatim}
This is ground control to Major Gubi, you've really made the grade!

                         !
                         !
                         ^
                        / \
                       /___\
                      |=   =|
                      |     |
                      |  G  |
                      |  U  |
                      |  B  |
                      |  I  |
                      |     |
                      |  X  |
                      |  X  |
                      |     |
                     /|##!##|\
                    / |##!##| \
                   /  |##!##|  \
                  |  / ^ | ^ \  |
                  | /  ( | )  \ |
                  |/   ( | )   \|
                      ((   ))
                     ((  :  ))
                     ((  :  ))
                      ((   ))
                       (( ))
                        ( )
                         .
                         .
                         .
	\end{verbatim}
		\subsection{Decisões Tomadas}
		Já tinhamos feito uma considerável parte da segunda fase na anterior, como o laço de execução e
		lidar com o input do usuário. Primeiramente nessa segunda parte começamos tirando os bugs residuais
		da primeira fase (tiros explodiam sem colidir com nada, algumas coisas não estavam sendo deletadas, ...) 
		e adicionando novas coisas que estavam faltando (nave e tiros colidem com inimigos, gravidade afeta os
		tiros, ...).\\
		
    		Percebemos que estava muito difícil jogar o jogo e perceber se o que estava acontecendo estava
		correto sem nenhum feedback além de texto dizendo ``o tiro colidiu com o inimigo''. Logo decidimos já começar
		fazer parte da visualização 3D do jogo.\\
		
    		Inicialmente utilizamos um bule 3D para testar o OpenGL. Posteriormente colocamos a nave, tiros e inimigos
		na tela, com as colisões funcionando, e já com uma implementação de mouse e teclado para o input do usuário.\\
    		
    		Decidimos deixar o movimento da nave independente da câmera do jogo, ou seja ela pode se mover
		nos eixos X e Y sempre orientada para frente e com velocidade constante. A câmera, controlada pelo mouse,
		como definido na parte 1, muda a orientação da ``arma da nave'', ou seja a orientação que os tiros serão 				disparados.\\
		
		\subsection{Problemas}
		Poder controlar a camera com o mouse foi um desafio, no começo ficou bem ruim, a distância da câmera à nave
		não era constante e parecia ruim, mas com um pouco de trabalho conseguimos melhorar, apesar de ter muito 				espaço para aperfeiçoar. \\
    		
    		Para imprimir texto na tela e ter uma interface adequada para o usuário, também tivemos várias complicações.
    		Ainda não conseguimos colocar um HUD mostrando a vida atual, pontuação, etc. Mas é uma de nossas prioridades.\\
    		
    		Prever para onde os inimigos deveriam atirar também não foi fácil. Não nos demos bem com o sistema de 				coordernadas do OpenGL, pois estavamos mais acostumados em mexer com 2D, a velocidade da nave deixa mais 				dificil, e a gravidade complica tudo ainda mais.
		
		
		\subsection{Próxima Parte}
		Para a terceira parte do projeto, planejamos polir mais o jogo. Deixar os gráficos melhores (os cubos
		atualmente dificultam a percepção de distância e orientação), colocar mais conteúdo (power ups, mais 					inimigos, mais tiros, e o que mais vier a nossa cabeça), arrumar bugs e deixar o jogo mais fluído.\\
		
		Também planejamos comentar mais o código que já existe e o código que ainda vamos criar.
		
		
	\newpage	
	\section{Terceira Fase}
	\begin{verbatim}
Your circuit's dead, there's something wrong
Can you hear me Major Gubi?
Can you hear me Major Gubi?
Can you hear me Major Gubi?
Can you...             
                         ___---___
                      .--         --.
 *                  ./   ()      .-. \.
                   /   o    .   (   )  \  *                *
         *        / .            '-'    \
                 | ()    .  O         .  |
                |                         |
                |    o           ()       |
                |       .--.          O   |
                 | .   |    |            |
                  \    `.__.'    o   .  /
                   \                   /
      *             `\  o    ()      /'       *
                      `--___   ___--'
                    *       ---

	\end{verbatim}
  \subsection{Decisões Tomadas}
        Já tinhamos avançado bastante na fase na anterior, já que tinhamos uma base gráfica e a maior parte dos elementos do jogo já prontas. Novamente começamos a terceira parte tirando alguns bugs e melhorando partes do código para melhor leitura e entendimento, dessa vez adicionando os elementos finais que montam nosso jogo. \\
    
        Para melhor polir o jogo, trocamos todos os modelos simples de ``caixa'' para modelos feitos no Blender, que apesar de um pouco toscos, dão um visual mais característico ao projeto (apesar dos prismas na parte anterior terem mais cara de um jogo ``clássico'').\\
    
        A parte de impressão ficou mais clara, e depois de vários desafios de fazer um texto imprimir na tela do jogo, conseguimos tal feito. Informações essências para o jogador como ``continues'' restantes, vida atual e pontuação foram adicionadas, assim como um texto para a tela de pause e de ``fim de jogo''.\\
        
        Junto dessas implementações melhoramos a câmera do jogo para uma que melhor impactava o gameplay do jogo. Foram feitos vários balanceamentos até encontrar uma que fosse razoavelmente melhor em comparação à anterior. A câmera fica centralizada com a nave, enquanto o mouse movimenta a mira e consequentemente a câmera.\\
    
        Um dos pontos cruciais do projeto foi deixar o projeto mais cross-platform. Antes apenas sistemas Linux tinham um Console para verificar o \emph{stdout}, após darmos uma fuçada no WinAPI, nós adicionamos também um console para Windows. Além disso, o método de compilação também tornou-se mais flexível, com tanto os comandos de compilação (\emph{make all}) como de limpeza (\emph{make clean}) funcionarem independentemente do sistema operacional. Isso poupou muito o tempo, além de se mostrar essencial ao debugarmos o projeto (afinal debugar sem \emph{stdout} fica difícil).
    
    \subsection{Problemas}     
        Também foi um pouco trabalhoso implementar o texto, já que o OpenGL não tem quase nenhum suporte nativo a texto, e mostrar texto 2D quando se está trabalhando em um ambiente 3D não é tão simples. \\

        Texto em 3D funcionou parcialmente. Algumas transformações como glTranslated funcionavam. Mas glScaled por alguma razão fazia com que o texto andasse para ``trás'' com uma velocidade dependendo de sua escala. Por exemplo, glScaled(1, 1, 1) funcionava perfeitamente. O texto ficava na posição desejada. Porém ficava enorme na tela. Ao escalarmos para glScaled(.8, .8, .8), o texto movia-se lentamente em direção a câmera. Não conseguimos descobrir o porquê nem conseguimos arrumar este problema. Por essa razão nós abandonamos o texto em 3D. Uma nota importante é que, por meio da classe TextBox (em específico a variação TextBox3D), quando em conjunto com a classe Item, produzia o texto 3D perfeitamente. Na posição correta e com escala gScaled(.1, .1, .1). No entanto, ao utilizarmos o texto em outras ocasiões, o problema aparecia. \\

        Iluminação foi a parte mais difícil, as luzes não funcionavam como queriamos e os modelos pareciam não ter profundidade. \\

        Testamos vários tipos de fontes de luz, posições e intensidades. No entanto, os objetos continuavam parecendo uma projeção 2D do modelo. O mais perto que chegamos foi mudar a intensidade da luz e posição de tal forma que a cor dos modelos mudavam para uma coloração mais clara. Porém, continuavam parecendo não ter profundidade. Ao final desistimos de colocar fontes de luz.\\
    
    \subsection{Próxima Parte}
      	Planejamos talvez continuar esse projeto, melhorando o que já fizemos e adicionando mais conteúdo para torna-lo ao menos ``jogável''. Gostamos de trabalhar com o OpenGL, apesar de ter sido bem sofrido a falta de classes ou de uma framework em volta do OpenGL que facilitasse o trabalho.
\end{document}
